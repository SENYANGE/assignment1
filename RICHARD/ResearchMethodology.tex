
\documentclass[10pt,]{article}
\usepackage{zed-csp,graphicx,color}%from

\begin{document}
\begin{titlepage}
 \begin{figure}[h]
  \centerline{\small MAKERERE 
  \includegraphics[width=0.1\textwidth]{muk_log} UNIVERSITY}
\end{figure}
\centerline{COLLEGE OF COMPUTING AND INFORMATIC SCIENCES}
\paragraph{•}
\centerline{DEPARTMENT OF COMPUTER SCIENCE\\}
\paragraph{•}

\centerline{COURSEWORK: RESEARCH METHODOLOGY(BIT 2207)\\}
\paragraph{•}

\centerline{LECTURER: MR.ERNEST MWEBAZE}
\paragraph{•}

\centerline{TOPIC\\} WAYS OF ERADICATING ENVIRONMENTAL POLLUTION FROM AUTOMOBILES IN LOW DEVELOPING COUNTRIES  BY INVENTING NONE FUEL CONSUMING  AUTOMOBILES.\\
 \centerline{CASE STUDY: UGANDA\\}
\paragraph{•}

%\centerline{\title{TOPIC\\ A RESEARCH REPORT ON THE RAPID INCREASE ON THE NUMBER OF BEDBUGS IN LUMUMBA HALL OF RESIDENCE}}
\centerline{PREPARED BY
 SENYANGE RICHARD}
 \paragraph{•}

\centerline{STUDENT NUMBER : 216001192.}\
\paragraph{•}
\centerline{REGISTRATION NUMBER:16/U/1102}
\paragraph{•}

%\maketitle



\end{titlepage}
\pagenumbering{roman}
\tableofcontents
\newpage
\pagenumbering{arabic}
\section{Introduction}

Pollution from automobiles for example motor vehicles is the single largest source of air pollution emissions. Automobiles exhaust is a complex mixture, composition of which depends on fuel used, and type and operating condition of the engine – whether it uses any pollution control devices. At present, motor fuels consists of Petrol, Diesel, LPG (mostly Butane) and CNG. In recent times, people have been very much successful in reducing motor vehicle pollutants; but due to enormous growth in population of vehicles on urban roads in developing countries, the effectiveness of the new technology in reducing pollution is not very much relevant and practicable. Over the year, engine efficiency has also gradually improved with progress in Electronic ignition, Fuel injection systems and Electronic control unit; and so, the emission standards however this alone isn’t enough; clean and none gas producing power is required if we’re to eliminate automobile pollution completely.
The study is aimed at inventing a self reproducing power  to replace fuel engine produced power. 

\section{Statement of the problem}
The purpose of this study is to identify the effect that environmental pollution from fuel using automobiles like cars and among others; have on the low developing countries .The basic hypothesis is that low developing countries ,are like the dust bins where developed countries dump the non wanted automobiles at a certain cost. 
\section{Scope of study}
The study focuses on the elimination of junk automobiles that developed countries send to low developing countries that emit disastrous carbonic gases that have endangered the lives of not only the people but also both animals and vegetation.
In the aspect that poisonous rainfall is received ,people are getting lung cancer ,animals are contracting weird diseases.

\section{Significance of study}
Socially the study is owed to enhance life of all living creatures not only in developing countries but also in other countries that produce these fuel automobiles.
In terms of resource utilization and depletion the study is meant to protect and conserve the environment  by limiting fuel mining that is disastrous .Advancing  technology  and finding other sources of power for example solar power, power from non fuel machines like motors ,turbines and among others.

\section{Specific objectives}


Creating automobiles that don’t use fuel as a power resource 
	Advancing technology and its love among the youth that are so optimistic and enthusiastic ,those that are creating power prototypes that produce power with fuel.
Reducing environmental pollution by generating systems that filter out those automobiles that are supposed to be used on roads and factories and those that are not basing on there pollution rate.
 
\section{Research methodology}
Quantitative methodology was used where I sampled a group of people in Uganda which is among the low the developing country where out of ten people who own automobiles(cars) at least, six of them their automobiles do give out disastrous exhaust fumes.
 
 \section{Recommendations}
The governments should set up some funding facilities to motivate the inventors to go down to work in the quest for clean powered automobiles to eradicate pollution completely.
The people should as well be synthesized on the effects of having and using fuel consuming automobiles that produce smoke from their  exhaust pipes like a fire place chimney.
\section{References}
http://environmentengineering.blogspot.com/2008/03/factors-to-be-considered-for.html\\

http://environmentengineering.blogspot.com/2008/03/benefits-and-shortcomings-of-battery.html\\

http://www.nhtsa.dot.gov/cars/rules/rulings/cafe/alternativefuels/availability1.htm\\

http://environmentengineering.blogspot.com/2008/03/options-of-various-alternative-fuels.html\\

http://environmentengineering.blogspot.com/2008/03/environment-friendly-hydrogen-gas-as.html

\end{document}






